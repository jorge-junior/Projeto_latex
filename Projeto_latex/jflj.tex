\documentclass[12pt]{article}
\usepackage[utf8]{inputenc}

\usepackage{graphicx}
\usepackage{natbib}
\usepackage{indentfirst}

\title{IF674 - Infra-estrutura de Computadores}
\author{Jorge Francisco}
\date{Dezembro 2021}

\begin{document}
\maketitle

\section{Introdução}
    A computação utiliza de diversos aparelhos a fim de realizar diversas tarefas de maneira eficiente e consistente, com isso é inegável que essa área para a computação e que, com os anos a mesma tem apresentado avanços de maneira quase que exponencial. Por isso a disciplina de infraestrutura de software, que trabalha os conceitos básicos dessa área estudando todos os componentes que formam um computador e com isso compreender seu funcionamento para assim obter um melhor desempenho, além de estudar formas de implementação. 
    \citep{wikiufpe}
    
    \begin{figure}[h!]
     \centering
     \includegraphics[scale=0.2]{image/componentes.jpg}
     \caption{componentes de um computador}
     \label{fig:componentes}\cite{imagem1}
    \end{figure}

\section{Relevância}
    Nos dias atuais, os computadores e equipamentos eletrônicos estão cada vez mais complexos e avançados, como por exemplo computadores de mesa, servidores e sistemas embarcados. Com isso, a compreensão dos limites de cada dispositivos e de seu funcionamento permite que o profissional possua um desempenho melhor na resolução de problemas relacionados ao hardware. 
    \citep{link}

\section{Relação com outras diciplinas}
    Essa disciplina possui relações de forma mais próxima com duas outras disciplinas que são infraestrutura de software e sistemas digitais. 
    
    \subsection{Infraestrutura de Software}
        Infraestrutura de software trabalha com o funcionamento da infraestrutura dos aplicativos dentro de um sistema desde jogos até browsers Web, nos quais interagem diretamente com o hardware já que determinam seu funcionamento como um conjunto. \citep{wikiufpe}
    \subsection{Sistemas digitais}    
        Essa disciplina trabalha com circuitos lógicos digitais onde são estudados circuitos de diversos níveis de complexidade.\citep{wikiufpe} 
    
\bibliographystyle{plain}
\bibliography{references}
\end{document}

\end{document}
